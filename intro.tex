\chapwithtoc{Úvod}
Současný vývoj softwarových projektů probíhá v~krátkých iteracích,
  aby mohl pružně reagovat na změny v~zadání.
Stejně tak i vývoj strojových překladačů probíhá v~krátkých iteracích,
  aby bylo možné pravidelně ověřovat zlepšení nebo zhoršení kvality překladů.
Z~toho plyne potřeba vývojářů testovat změny,
  které provedli v~rámci iterace,
  která mohla trvat 30 sekund, hodinu, den nebo týden.
K~testování funkčnosti softwarových projektů lze použít různé sady testů -- od unit testů po akceptační testy.
Je možné stejným způsobem testovat i kvalitu strojových překladačů?

Je samozřejmé, 
  že vývojáři strojových překladačů testují svůj kód stejně jako ostatní vývojáři.
Pomocí těchto testů mohou zajistit,
  aby překladač fungoval,
  ale nemohou takto jednoduše zajistit kontrolu kvality překladů.
Kontrolu překladů může provádět sám vývojář,
  ale ztratí tím spoustu času a energie,
  kterou by mohl lépe využít při dalším vývoji.

Žádná kontrola kvality, kterou není možné automatizovat,
  vývojářům nepomůže zrychlit vývojový cyklus.
Proto je třeba,
  aby existoval způsob jak rychle a opakovaně kontrolovat kvalitu překladů.

Proto byly vynalezeny metriky strojového překladu,
  které umožňují rychle a opakovaně testovat překlady.
Tyto metriky se snaží hodnotit překlady tak,
  aby jejich výsledky co nejvíce odpovídaly lidskému hodnocení.

Avšak výsledek většiny metrik je pouhé jedno číslo.
Je třeba, aby vývojaři měli k~dispozici nástroje,
  pomocí kterých budou moci efektivně vyhodnocovat své překlady.
Pomocí takovýchto nástrojů si vývojáři mohou vyhodnotit,
  co způsobily změny, které provedli.

Vývojář by měl mít možnost překlady nejen vyhodnocovat,
  ale i porovnávat.
Porovnání dvou různých překladů může totiž vývojáři objasnit jevy,
  o~kterých nemusel vůbec vědět.
A~na základě takto získaných poznatků pak může vyvinout novou lepší verzi překladače.
A~po mnoha iteracích vývoje snad vyvine překladač,
  s~kterým bude moci být spokojený.

Na trhu existuje několik nástrojů,
  které umožňují vyhodnocovat překlady pomocí různých metrik
  nebo porovnávat překlady na základě různých kritérií.
Důležité je,
  aby mohl vývojář použít přístup,
  který mu vyhovuje.

Nástrojem,
  který umožňuje vývojářům efektivně porovnávat a vyhodnocovat překlady,
  se snaží být \mbox{MT-ComparEval}, 
  o~němž pojednává tato bakalářská práce.
