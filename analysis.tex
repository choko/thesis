\chapter{Analýza}
Vývoj nástroje MT-ComparEval byl

\section{Funkční požadavky}
Hlavní požadavek na funkcionalitu byl,
  aby nástroj bylo možné jednoduše spouštět na vývojářově počítači.



\section{Podobné aplikace}
Vývoj nástroje MT-ComparEval byl v určitých oblastech inspirován nástroji,
  které už mohou vývojáři běžně využívat při své práci.
Z těchto nástrojů jsme se snažili vybrat ty nejdůležitější vlastnosti,
  které jsme později zkombinovali do jednoho funkčního celku.
V následující části budou tyto nástroje podrobněji představeny.

\subsection{mteval-11b.pl}
Je skipt napsaný v jazyce Perl, který umožňuje počítat metriky BLEU a NIST.
Tento skript je celosvětově používaný,
  a proto byl použit pro kontrolu,
  že metriku BLEU v MT-ComparEval počítáme správně.

V současné době už existuje verze mteval-13a.pl,
  v které mimojiné přibyla možnost počítat metriky pro jednotlivé segmenty v překladu.
Postup, který k tomu použila je dále analyzován v kapitole Metriky strojového překladu.

\subsection{iBLEU}

\subsection{EMS - An Experimental Management System}
