\chapter{Motivace}
V této kapitole budou v krátkosti představeny všechny důvody,
  proč a jak byl vyvinut nástroj MT-ComparEval.
Všechny důvody pak budou více rozebrány v~následujících kapitolách.

\section{Experimenty a tasky}
Při vyhodnocování strojových překladů porovnání s referenčním překladem
  (dále budeme používat pouze reference).
Strojové překladače mohou být zaměřeny na různé domény textu (např. novinové články, beletrie apod.) 
  nebo na různé jazykové páry (např. z angličtiny do češtiny, z němčiny do češtiny apod.).
Aby uživatelé nástroje MT-ComparEval mohli snadno vyhodnocovat své strojové překladače na různých doménách nebo pro různé jazykové páry,
  mohou si vytořit různé \uv{experimenty},
  v rámci kterých mohou porovnávat své překlady s příslušnými referencemi.
Na Obrázku \ref{img:experiments} je vidět příklad vytvořených experimentů z různých domén a jazykových párů.
\begin{figure}
	\caption{Přehled vytvořených experimentů v nástroji MT-ComparEval}
	\label{img:experiments}
\end{figure}

V každém experimentu pak uživatel může vytvářet \uv{tasky},
  které později může vyhodnocovat a porovnávat.
Výpis tasků jednoho z experimentů je ukázán na Obrázku \ref{img:tasks}.
\begin{figure}
	\caption{Přehled vytvořených tasků v jednom z experimentů v nástroji MT-ComparEval}
	\label{img:tasks}
\end{figure}

O významu experimentů, tasků a jiných pojmů je možné se více dozvědět ve~\ref{chap:experiments}.~kapitole.


\section{Metriky strojového překladu}
Aby bylo možné jednotlivé překlady porovnávat,
  byly vynalezeny metriky 
V nástroji MT-ComparEval jsou metriky počítány na úrovni celých tasků 
  (na Obrázku \ref{img:compare_metrics_tasks} je vidět porovnání metrik dvou tásků)
  nebo na úrovni jednotlivých vět
  (Obrázek \ref{img:compare_metrics_sentences} ukazuje porovnání metrik pro dva různé překlady).
Metrikami strojových překladů se více zabývá \ref{chap:metrics}. kapitola.
V té budou představeny všechny metriky,
  které jsou použity v nástroji MT-ComparEval.

\begin{figure}
	\caption{Porovnání metrik dvou tasků v nástroji MT-ComparEval.}
	\label{img:compare_metrics_tasks}
\end{figure}

\begin{figure}
	\caption{Porovnání metrik dvou překladů v nástroji MT-ComparEval.}
	\label{img:compare_metrics_sentences}
\end{figure}


\section{Porovnávaní dvou strojových překladů}
Kvalitu strojových překladačů je možné vyhodnotit i porovnáním jednotlivých vět.
Nástroj MT-ComparEval umožňuje procházet věty seřazené podle metriky strojových překladů 
  a hledat v těchto větách rozdíly,
  které ovlivnily výsledné metriky.
Aby uživatelé mohli hledat rozdíly mezi větami,
  je možné zobrazit potvrzené n-gramy (n-gramy, které se nacházeji v referenci i strojovém překladu),
  zlepšující n-gramy (potvrzené n-gramy, které se nacházejí pouze v jednom z porovnávaných překladů)
  nebo zhoršující n-gramy (nepotvrzené n-gramy, které se nacházejí pouze v jednom z porovnávaných překladů).
Na Obrázku \ref{img:compare_sentences} je vidět porovnání dvou překladů se zvýrazněnými n-gramy.

Strojové překlady nemusejí být porovnávány pouze na základě strojových metrik,
  další informací,
  díky které je možné si udělat lepší představu o vlastnostech strojového překladače,
  jsou přehledy nejvíce zlepšujících a zhoršujících n-gramů v jednotlivých překladech.
Nástroj MT-ComparEval nabízí přehled nejvíce zlepšujících a zhoršujících n-gramů
  v jednotlivých překladech.
Takový přehled je možné vidět i na obrázku \ref{img:confirmed_ngrams}.

Přehledy zlepšujících i zhoršujicích n-gramů mohou byt použity k filtrování vět,
  aby si uživatel mohl snadno prohlédnout věty,
  ve kterých se dané n-gramy nacházejí.
Na Obrázku \ref{img:filtered_sentences} je vidět výpis vět vyfiltrovaných podle zlepšujícího n-gramu.


\begin{figure}
	\caption{
		Porovnání dvou překladů v nástroji MT-ComparEval.
		Pastelovými odstíny žluté a modré barvy jsou zvýrazněny potvrzené n-gramy,
		sytými odstíny žluté a modré barvy jsou zvýrazněny zlepšující n-gramy
		a červenou barvou jsou zvýrazněny zhoršující n-gramy.
	}
	\label{img:compare_sentences}
\end{figure}

\begin{figure}
	\caption{Přehled nejvíce zlepšujících n-gramů v jednotlivých překladech.}
	\label{img:confirmed_ngrams}
\end{figure}

\begin{figure}
	\caption{Výpis vět, v kterých se nachází zlepšující n-gram ??}
	\label{img:filtered_sentences}
\end{figure}

O algoritmech,
  které byly použity při porovnávání dvou překladů a hledání pozic potvrzených n-gramů,
  pojednává \ref{chap:compare}. kapitola.


\section{Automatizace a snadné použití}
Nástroj MT-ComparEval byl navržen tak,
  aby bylo možné jeho použití co nejvíce automatizovat.
Uživatelé tak nemusí ručne vytvářet každý task,
  ale mohou si napsat jednoduché skripty,
  pomocí nichž mohou vytvářet tasky automaticky při každém commitu, buildu,\dots

V \ref{chap:users}. kapitole je možné najít informace,
  jak se používá nástroj MT-ComparEval
  a jak je možné nasadit ho do vývojového procesu.

\section{Rozšiřitelnost}
Každý uživatel může preferovat různé metriky pro vyhodnocování strojových překladů.
V nástroji MT-ComparEval jsou předprogramovány pouze některé.
  avšak uživatel si může snadno doimplementovat svoje vlastní.

O tom, jak si uživatel může doprogramovat vlastní metriky nebo jak byl nástroj MT-ComparEval vyvinut,
  je možné se dočíst v \ref{chap:programmers}. kapitole.

