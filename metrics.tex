\chapter{Metriky strojového překladu}
Při porovnávání dvou strojových překladů je potřeba,
  abychom byli schopní snadno a rychle určit,
  jak jsou jednotlivé překlady dobré.
Posuzování kvality překladů člověkem je časově náročné
  a není možné ho automatizovat.
Proto byly vytvořeny metriky,
  jejichž výsledky jsou podobné výsledkům získaných od lidí,
  ale je možné je rychle spočítat,
  aplikovat na různé jazyky a opakovat dle libosti.
Mezi tyto metriky patří i metrika BLEU,
  která je zvolena jako výchozí metrika v nástroji MT-ComparEval.

Při počítání metrik strojových překladů se snažíme zjistit,
  jak moc se strojový překlad podobá překladu od člověka - tzv. referenčnímu překladu.
Čím více se podobá strojový překlad překladu referenčnímu,
  tím je lepší.
To ovšem neznamená, že překlad s nižší hodnotou metriky je špatný.
Překlad totiž můžeme provést několika různými způsoby 
  ( např. můžeme změnit slovosled, použít synonyma, \dots ),
  z nichž se pouze jeden bude shodovat s referenčním překladem.
Tento problém může zmírnit použití více referenčních překladů,
  ale úplně odstranit ho nelze.
Některé překladatelské systémy jsou optimalizovány pro určité metriky,
  avšak jejich výsledný překlad nemusí být ideální.
Proto je třeba,
  abychom při vyhodnocování strojového překladu nespoléhali pouze na jednu metriku.

\section{BLEU}
Nejvíce používanou metrikou pro vyhodnocování překladů je metrika BLEU.
Hlavním důvodem je to,
  že hodně odpovídá lidskému vyhodnocení překladu
  a je možné ji rychle spočítat.
Rychlost výpočtu metrik je pro vývojáře překladových systémů velmi důležitá,
  protože vývojáři potřebují každou změnu v systému otestovat,
  aby si mohli být jisti,
  že překladový systém nepokazili.

Při počítání metriky BLEU počítáme precision n-gramů,
  což nám říká,
  kolik n-gramů z vyhodnocovaného překladu se nachází i v referenčním překladu.	
$$ \text{PRECISION} = \frac{\lvert \lbrace \text{n-gramy z referenčního překladu} \rbrace \cap \lbrace \text{n-gramy z vyhodnocovaného překladu} \rbrace \rvert}{\lvert \lbrace \text{n-gramy z vyhodnocovaného překladu} \rbrace \rvert}$$

Pří hledání potvrzených n-gramů si musíme dát pozor na překlady,
  ve kterých se nachází více stejných n-gramů než je daných n-gramů v referenčním překladu.
V takovém případě vezmeme vždy jen
  $min( \lvert \lbrace \text{n-gramy z referenčního překladu} \rbrace \cap \lbrace \text{n-gramy z vyhodnocovaného překladu} \rbrace \rvert, \lvert \lbrace \text{n-gramy z referenčního překladu} \rbrace \rvert )$
V článku o BLEU pomocí takto nalezených potvrzených n-gramů počítají modified n-gram precision:
$$ \text{MODIFIED PRECISION} = \frac{ \lvert \text{potvrzené n-gramy} \rvert }{ \lvert \text{n-gramy z vyhodnocovaného překladu} \rvert } $$




\section{BLEUS}
Abychom mohli počítat metriku BLEU i pro jedntolivé věty,
  musíme se vypořádat s tím,
  že se v některých větách nemusí vyskytovat žádné potvrzené 1-gramy, 2-gramy, 3-gramy nebo 4-gramy.
V tom případě by výsledná metrika těchto vět byla rovna nule,
  i když by se v nich nějaký potvrzený n-gram nacházel.
Tento problém řeší vylepšení metriky BLEU - metrika BLEUS. %% přidat citaci
Ta připočítává +1 ke všem n-gramům delším než jedna.
Tím pádem můžeme spočítat metriku i pro věty,
  které nemají potvrzené n-gramy libovolné délky.
Zároveň dokážeme ohodnotit úplně špatný překlad nulovou metrikou,
  protože 1-gramy vyhlazovány nejsou.

Skript mteval-13a.pl, %% přidat citaci
  který je ve světě považován za autoritu pro vyhodnocování překladů,
  používá jiný způsob vyhlazování.
Místo přičítání +1 ke všem n-gramům má speciální vzoreček pro počítání precision o chybějících potvrzených n-gramů.
Precision je rovna $1 / 2^k$, kde $k = 1$ u první n takové,
  že v překladu není žádný potvrzený n-gram.

Rozdíl obou metrik si můžeme demonstrovat na větě délky 4,
  která obsahuje pouze jeden potvrzený 2-gram
  (z čehož plyne, že obsahuje i dva potvrzené 1-gramy).

\begin{tabular}{| l | c | c || c | c |}
\hline
& & & \multicolumn{2}{c|}{PRECISION} \\
\cline{4-5}
délka & počet potvrzených & celkový počet & BLEUS & mteval \\
\hline
1-gram & 2 & 4 & $\frac{2}{4} = 0.50$ & $\frac{2}{4} = 0.50$ \\
2-gram & 1 & 3 & $\frac{1+1}{3+1} = 0.50$ & $\frac{1}{3} \approx 0.33$ \\
3-gram & 0 & 2 & $\frac{0+1}{2+1} \approx 0.33$ & $\frac{1}{2^1} = 0.50$ \\
4-gram & 0 & 1 & $\frac{0+1}{1+1} = 0.50$ & $\frac{1}{2^2} = 0.25$ \\
\hline \hline 
\multicolumn{3}{ |r| }{bleu} & $\frac{1}{2^3 3}$ & $\frac{1}{2^4 3}$ \\
\hline
\end{tabular}

To, že se metriky liší,
  nám vůbec nevadí,
  protože obě metriky ve většině případů dokáží rozpoznat lepší překlad od horšího.

Problém, se kterým se vyhlazování u obou přístupů nedokáže vypořádat, je,
  když porovnáváme dva překlady, z nichž jeden je kratší než 4 slova. 
Vyhlazovaní u mteval používá výjimku u precision n-gramů,
  kde n je větší než počet slov ve větě. 
U těchto n-gramů nastavuje precision rovno jedné.
Krátší překlad, který může být horší,
  pak bude vyhodnocen jako lepší než skutečně lepší překlad.

Chybu si můžeme ukázat na dvou větách.
První věta dlouhá dvě slova nebude obsahovat žádné potvrzené n-gramy.

\begin{tabular}{| l | c | c || c | c |}
\hline
& & & \multicolumn{2}{c|}{PRECISION} \\
\cline{4-5}
délka & počet potvrzených & celkový počet & BLEUS & mteval \\
\hline
1-gram & 1 & 2 & $\frac{1}{2} = 0.50$ & $\frac{1}{2} = 0.50$ \\
2-gram & 0 & 1 & $\frac{0+1}{1+1} = 0.50$ & $\frac{1}{2^1} = 0.50$ \\
3-gram & 0 & 0 & $\frac{0+1}{0+1} = 1$ & 1 \\
4-gram & 0 & 0 & $\frac{0+1}{0+1} = 1$ & 1 \\
\hline \hline 
\multicolumn{3}{ |r| }{bleu} & $\frac{1}{2^2}$ & $\frac{1}{2^2}$ \\
\hline
\end{tabular}

Jako druhou použijeme větu z předchozího příkladu,
  pro níž vyšlo vyhlazené BLEUS$ = \frac{1}{2^3 3}$
  a mteval$= \frac{1}{2^4 3}$.
U horšího překladu tak vyšlo BLEUS $6\times$ větší a mteval $12\times$ větší. 
Díky této nemilé vlastnosti vyhlazování BLEU se může stát,
  že při procházení vět nenajdeme takto špatný překlad,
  protože bude vyhodnocen jako lepší.

\section{Potvrzené n-gramy}
Abychom částečně napravili chyby popsané výše,
  můžeme jako metriku zavést počet potvrzených n-gramů v překladu.
Touto metrikou bude vždy horší překlad označen za horší,
  ale problém nastane při řazení podle absolutního rozdílu,
  protože v delších větách může být více potvrzených n-gramů,
  budou většinou dlouhé věty předcházet kratší věty,
  u kterých se správnost překladu mohla mnohem více lišit.

\section{Recall}
Pří výpočtu BLEU se počítá precision n-gramů ze strojového překladu.
Pokud místo n-gramů ze strojového překladu použijeme n-gramy z referenčního překladu,
  vyjde nám metrika Recall,
  která říká s jakou pravděpodobností se nachazí n-gram z referenčního překladu i v překladu strojovém.
$$ RECALL = \prod_{l=1}^{4} \frac{ \lvert \text{potvrzené n-gramy délky l} \rvert }{ \lvert \text{referenční n-gramy délky l} \rvert } $$

Touto metrikou, případně s vyhlazením stejným jako v BLEUs,
  budeme moci vyřešit oba dva výše zmíněné problémy - 
  t.j. problém s krátkým překladem a problém s řazením podle potvrzených n-gramů.

Nicméně i Recall má své problémy.
Strojové překlady,
  které jsou výrazně delší než překlady referenční,
  by mohly mít vysokou hodnotu Recall,
  ale i tak by se jednalo o špatné překlady,
  protože by obsahovaly mnoho nadbytečných slov. 
Tento problém řeší metrika F-Measure,
  která bude vysvětlena v následující části.

\section{F-Measure}
Metrika F-Measure kombinuje Recall i Precision,
  čímž zajišťuje, 
  že zbytečně dlouhé překlady budou mít horší výsledky.

$$ \text{F-MEASURE} = 2 \cdot \frac{\text{PRECISION} \cdot \text{RECALL}}{\text{PRECISION} + \text{RECALL}} $$

\section{Diskuse}
