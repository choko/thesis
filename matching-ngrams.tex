\newtheorem*{define}{Definice}	% Definice nečíslujeme, proto "*"
\newtheorem*{example}{Příklad}	% Příklady nečíslujeme, proto "*"

\chapter{Zvýrazňujeme výsledky porovnání dvou systémů}
Při vyhodnocování strojových překladů nás zajímá,
  která slova a slovní spojení byly přeloženy správně (potvrzené n-gramy)
  a které správně přeloženy nebyly (nepotvrzené n-gramy).
Při porovnávání dvou překladových systémů chceme zvýraznit slova a slovní spojení,
  které byly přeloženy:

\begin{itemize}
  \item Správně v obou porovnávaných systémech \\
    Tato slova nepřispěla k zlepšení ani k zhoršení BLEU žádného
    z porovnávaných překladových systémů.
  \item Správně v jednom z porovnáváných systémů \\
    Tato slova přispěla k zlepšení BLEU překladového systému
  \item Špatně pouze v jednom z porovnávaných systémů \\
    Tato slova přispěla k zhoršení BLEU překladového systému
\end{itemize}

\section{Hledáme pozice potvrzených n-gramů}


Nejjednodušší práci máme,
  pokud je počet výskytů jednotlivých n-gramů ve strojovém překladu
  menší nebo roven počtu výskytů odpovídajících n-gramů v referenčním překladu.
Pak můžeme všechny n-gramy,
  které mají alespoň jednu podoporu v referenčním překladu,
  zvýraznit jako n-gramy potvrzené.

Horší situace nastane,
  pokud bude počet výskytů některého n-gramu ve strojovém překladu
  vysšší než počet výskytů odpovidajících n-gramů v referenčním překladu.

\subsection{Vybíráme si z několika možností}
Při vyhodnocování strojových překladů můžeme narazit na situaci,
  kdy se ve strojovém překladu nachází více výskytů n-gramu než v referenčním překladu.
Naším úkolem tedy je, abychom zvýraznili výskyty n-gramů,
  které nejvíce odpovídají výskytům n-gramů v referenčním překladu.

\begin{example}
{\sl RT} A , B C\\
{\sl MT} A , B , C \\
\end{example}

\begin{define}
{\sl Nejdelší společná podposloupnost(LCS)} posloupností A a B, je posloupnost,
  kterou můžeme získat z posloupností A a B odstraněním některých prvků. 
\end{define}

Výskyty n-gramů, které se nacházejí v nejdelší společné podposloupnosti,
  jsou z definice LCS 

\subsection{Odhalujeme špatný slovosled}
Další problém, na který můžeme často narazit, je změna slovosledu.
S tou si metoda používající global alignment neporadí. 

\begin{example}
{\sl RT} A B C D \\
{\sl MT} A C B D \\
\end{example}

\subsection{Nevýhody tohoto řešení}

