\chapter{Uživatelská dokumentace}
\label{chap:users}

\section{Systémové požadavky}
Nástroj \mbox{MT-ComparEval} byl vyvinut jako webová aplikace,
  která poběží na serveru s~Linuxem.\footnote{
    Pokud je aplikace nainstalována na vzdáleném serveru, může být používána pomocí libovolného moderního webového prohlížeče na libovolném operačním systému.
  }
Dále musí být na uživatelském počítači nainstalované PHP 5.4 a databáze SQLite 3.

\section{Instalace a spuštění programu}
Instalace není vůbec náročná,
  stačí spustit script \textbf{./bin/install.sh},
  který připraví vše potřebné pro běh programu. 
Jelikož chceme,
  aby uživatel nemusel používat webserver Apache nebo jiné webservery,
  musí si uživatel před každým spuštěním aplikace zapnout lokální server pomocí skriptu \textbf{./bin/server.sh},
  který spustí aplikaci na adrese \textbf{http://localhost:8080}.
Poslední krok spočívá ve spuštění programu,
  který kontroluje nově přidané experimenty a tasky.
Ten se spustí pomocí skriptu \textbf{./bin/watcher.sh}.
Pro celkové zjednodušení spouštění aplikace byl přidán skript,
  který všechny tyto kroky sjednotí do jednoho,
  a proto je možné spustit celou aplikaci pomocí příkazu \textbf{./bin/run.sh}.

V~případě, že má uživatel nainstalován nějaký webový server
  a chce ho použít pro provoz nástroje \mbox{MT-ComparEval},
  může ho použít jako v~případě ostatních webových stránek,
  ale nesmí zapomenout spustit skript pro kontrolu nově přidaných experimentů a tasků.

\section{Import experimentů}
Aby uživatel mohl porovnávat překlady,
  musí nejprve vytvořit experiment,
  v~jehož rámci bude jednotlivé překlady vyhodnocovat.
Každý experiment je uložen ve vlastním podadresáři adresáře \textbf{./data},
  do kterého uživatel musí nahrát zdrojový text a referenční překlad.
Výchozí jména souborů jsou \textbf{source.txt} pro zdrojový překlad
  a \textbf{reference.txt} pro referenční překlad.
U~všech souborů s~větami se předpokladá,
  že každá věta či souvětí je na vlastním řádku
  a je zachováno pořadí vět.
To znamená, že první řádek v~souboru se zdrojovým textem je přeložen na prvním řádku souboru s~referenčním a strojovým překladem,
  druhý řádek v~souboru se zdrojovým textem odpovídá druhému řádku v souboru s referencí atd..
Pro správný běh aplikace je nutné, aby soubory se zdrojovým textem a referenčním překladem měly stejný počet řádků.
Pokud nebudou mít stejný počet řádků, experiment nebude importován.
Jako jméno experimentu je použito jméno adresáře.

\subsection{Konfigurace experimentu}
Výše zmíněný přístup je trošku omezený a nenabízí uživatelům možnosti konfigurace.
Proto je možné,
  aby si uživatel v~konfiguračním souboru \textbf{./config.neon} předefinoval výchozí hodnoty.
Je tak možné změnit jméno experimentu, popisek experimentu nebo jména souborů,
  ve kterých se bude hledat zdrojový text či referenční překlad.

Konfigurace experimentu by mohla vypadat například takto: \\

\begin{verbatim}
name: Nakonfigurovaný experiment
description: Ukázka konfigurace experimentu
source: zdrojovy_text.txt
reference: referencni_preklad.txt
\end{verbatim}

Pole name odpovídá názvu experimentu, které bude použito místo jména adresáře.
Description je jednořádkový popis experimentu,
  který odpovídá např. hlavičce commit message.
Source a reference jsou relativní jména souborů,
  ze kterých bude načten zdrojový text resp. referenční překlad.

Informace o~importu experimentu je možné nalézt
  jak v~lokálním logu každého experimentu v~souboru \textbf{./data/experiment/import.log},
  tak v~globálním logu všech importů v~souboru \textbf{./log/import.log}.
Z~toho logu uživatel může poznat,
  jestli byl daný experiment úspěšně importován,
  případně k~jakému problému při importu došlo.

V~případě, že se import nezdaří,
  může uživatel opravit chyby,
  kvůli kterým byl import přerušen,
  a znovu experiment nahrát do daného adresáře.
Musí však smazat soubor \textbf{./data/experiment/.notimported},
  který zabraňuje dalším importům u~experimentů,
  při jejichž importu došlo k~chybě.


\section{Import tasků}
Tasky jsou nahrávány jako podadresáře v~adresáři experimentu,
  s~jehož referenčním překladem má být daný překlad porovnán.
Jediný soubor, který musí být do tohoto podadresáře nahrán je soubor \textbf{translation.txt},
  v~kterém budou všechny přeložené věty.
Opět je nutné, aby soubor s přeloženými větami měl stejný počet řádků jako soubor s referenčním překladem.
Výchozí jméno tasku je stejné jako jméno podadresáře, v kterém se nachází.

\subsection{Konfigurace tasku}
Stejně jako je možné konfigurovat experiment, je možné konfigurovat i task.
Task má konfigurovatelné jméno (name), popisek (description),
  relativní cestu k~souboru s překladem (translation) a volbu,
  zda mají být předpočítány zlepšující a zhoršující \mbox{n-gramy} (precompute\_ngrams).
U~všech tasků, pokud uživatel neřekne jinak, 
  jsou tyto \mbox{n-gramy} předpočítány.
Konfigurace tasku může vypadat například takto:

\begin{verbatim}
name: Nakonfigurovaný task
description: Ukázka konfigurace task
translation: preklad.txt
precompute_ngrams: false
\end{verbatim}

Záznam o~průběhu importu je uložen v~adresáři tasku v~soboru \textbf{import.log},
  pomocí něhož mohou být odhaleny případné problémy při importu
  a následně mohou být odstraněny stejným způsobem jako tomu bylo při importu experimentů.

V~obou konfiguračních souborech není nutné konfigurovat všechny položky.
Pokud bude některá položka vynechána,
  bude použita její výchozí hodnota. 

\section{Webové prostředí}
Porovnávání tasků je realizováno pomocí webové aplikace,
  která při zapnutém lokálním serveru běží na adrese \textbf{localhost:8080}.
Uživatel si nejprve musí zvolit experiment,
  jehož tasky chce porovnávat, a následně si z~těchto tasků vybere dva k~porovnání.
%% vložit obrázek s výpisem experimentu
%% vlozit obrázek s výpisem tasku

V~případě,
  že uživatel již v~budoucnu nebude chtít používat daný experiment nebo task,
  může jej smazat pomocí příslušného odkazu ve webovém prostředí.

\subsection{Porovnávání tasků}
K~porovnání dvou tasků je možné přistoupit z~několika pohledů.
Každému pohledu odpovídá jedna záložka v~horním menu stránky s~porovnáním.
Kvalitu překladu tasků je možné posuzovat na základě překladu jednotlivých vět,
  vypočtených metrik nebo nalezených zlepšujících resp. zhoršujících \mbox{n-gramů}.
Při porovnávání si uživatel může zvolit metriku,
  podle níž se bude řídit výpis vět nebo grafů metrik.
Změnou metriky se tak mění pořadí vět,
  grafy s~rozdílem hodnot metriky na úrovni jednotlivých vět
  a grafem s~hodnotami z~párového bootstrap resamplingu.
Ve výchozím nastavení se věty zobrazují od nejvíce zlepšujících po nejvíce zhoršující,
  ale toto pořadí lze stejně jako metrika snadno změnit.\footnote{
    Míra zlepšení daného překladu je určena rozdílem aktivní metriky u porovnávaných překladů.
  }

V~případě potřeby je možné přímo v~porovnání měnit jaké tasky se mají porovnávat,
  což může být výhodné v~případě,
  že uživatele zajímá,
  jak by jeden z~porovnávaných tasků obstál v~porovnání s~jiným.
%% vložit obrázek s horní lištou

\subsubsection{Věty}
Záložka \textbf{Sentences} slouží k~zobrazení všech vět z~obou překladů.
U~každé věty se zobrazuje zdrojový text,
  referenční překlad, oba porovnávané překlady
  a metriky jednotlivých překladů.
Každou z~těchto informací je možné zobrazit/skrýt pomocí panelu \textbf{Options}.
Jak již bylo zmíněno dříve,
  věty jsou seřazeny podle rozdílu aktivní metriky,
  kterou je možné kdykoliv změnit.
Věty se po každé změně metriky načítají znovu,
  aby vždy byly správně seřazeny.
Načítání vět probíhá po částech,
  s~tím jak uživatel posouvá stránku,
  nové věty se načítají až ve chvíli,
  kdy se dostane na konec stránky.
%% vložit obrázek s výpisem vět

V~panelu \textbf{Options} je možné zapnout i zvýraznění potvrzených \mbox{n-gramů}, zlepšujících nebo zhoršujících \mbox{n-gramů}.
N-gramy jsou zvýrazněny pomocí barvy pozadí.
Všechny potvrzené \mbox{n-gramy} jsou zvýrazněny pastelovou barvou v~referenčním i porovnávaném překladu,
  zlepšující \mbox{n-gramy} jsou zvýrazněny v~porovnávaných překladech sytou barvou
  a zhoršující \mbox{n-gramy} jsou zvýrazněny pastelovou červenou barvou.
%% vložit věty se zvýrazněnými potvrzenými \mbox{n-gramy}
%% vložit věty se zvýrazněnými vylepšujícími \mbox{n-gramy}
%% vložit věty se zvýrazněnými zhoršujícími \mbox{n-gramy}

Kromě zvýraznění \mbox{n-gramů} lze zobrazit diff mezi referenčním překladem a jedním z~porovnávaných překladů.
Diff je zvýrazněn pomocí barevného podtržení jednotlivých slov,
  zelené podtržení znamená, že se podtržené slovo nachazí v~referenčním i porovnávaném překladu,
  a červené podtržení znamená, že se podtržené slovo nachází pouze v~jednom z~překladů.
%% vložit věty se zvýrazněným diffem

\subsubsection{Statistiky}
V~záložce \textbf{Statistics} může uživatel nalézt porovnání všech spočtených metrik pro oba tasky.
Zároveň s~tímto porovnáním se zde nachází i graf s~rozdíly hodnot metrik na úrovni jednotlivých vět,
  který určuje rozdělení těchto rozdílů.
Druhý graf, který se nachází na této záložce,
  je graf hodnot z~bootstrap resamplingu,
  na jehož základě je možné určit,
  který z~tasků je signifikantě lepší.
%% vložit obrázek s výpisem statistik

\subsubsection{Zlepšující a zhoršující \mbox{n-gramy}}
Na záložkách \textbf{Confirmed \mbox{n-grams}} a \textbf{Unconfirmed \mbox{n-grams}} jsou vypsány tabulky s~nejvíce zlepšujícími resp. zhoršujícími \mbox{n-gramy}.
Kliknutím na některý z~\mbox{n-gramů} jsou zobrazeny věty,
  ve kterých se daný zlepšující resp. zhoršující \mbox{n-gram} nachází.
%% vložit obrázek s vylepšujícími \mbox{n-gramy}
Tento \mbox{n-gram} je ve větách zvýrazněn černým rámečkem,
  aby bylo na první pohled patrné,
  kde se nachází.
Věty se zlepšujícími resp. zhoršujícími \mbox{n-gramy} nejsou řazeny podle vybrané metriky,
  ale počtu výskytu daného \mbox{n-gramu} ve větě.
%% vložit obrázek se zobrazeným vylepšujícím \mbox{n-gramem}
Uživatel se v~případě potřeby může snadno vrátit k~výpisu všech vět, pomocí příslušného odkazu.
%% vložit obrázek s volbou pro změnu metriky



