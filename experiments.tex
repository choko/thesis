\chapter{Názvosloví MT-ComparEval}
\label{chap:experiments}

Správné pochopení pojmů \textbf{experiment},\textbf{task}, \textbf{\mbox{n-gram}}, \dots je důležité k~pochopení fungování celého nástroje MT-ComparEval.
Proto v~této kapitole bude podrobněji vysvětleno,
  co tyto pojmy znamenají a jaké jsou mezi nimi vztahy.

\section{Experiment}
Uživatelé mohou chtít testovat své překladače na různých textových doménách různých délek
  nebo testovat své překladače pro různé jazykové páry.
Nástroj MT-ComparEval musí umožňovat vytvořit různá vyhodnocovací prostředí,
  ve kterých uživatelé mohou testovat své strojové překladače.
Takovéto testovací prostředí se v~nástroji MT-ComparEval nazývá \textbf{experiment}.

Každý experiment obsahuje vlastní \textbf{zdroj} (věty, které mají být strojovým překladačem přeloženy) a
  \textbf{referenci} (člověkem přeložené věty, které budou později použity k~vyhodnocování jednotlivých strojových překladačů).
Pomocí různých dvojic zdrojů a referencí mohou uživatelé vyhodnocovat strojové překladače na různých testovacích sadách,
  což se jim může hodit v~případě,
  kdy se překladač na různých zdrojích chová různě.

\section{Task}
V~rámci experimetu uživatelé mohou chtít porovnávat různé překlady zdrojových vět.
Ať už se jedná o~různé verze jednoho strojového překladače nebo různé strojové překladače.
Aby mohl uživatel vyhodnocovat různé překlady,
  umožňuje nástroj MT-ComparEval nahrávat do experimentů tzv. \textbf{tasky}.
Každý task reprezentuje jednu verzi překladu zdrojových vět,
  která později může být porovnána s~jinou verzí překladu.
V~rámci této bakalářské práce budou strojové překlady nazývány zkráceně slovem \textbf{překlady}.

O~vytváření experimentů a tasků i o~dalších možnostech použití nástroje MT-ComparEval je možné se dozvědět v~\ref{chap:users}. kapitole.

\section{N-gramy}
Při vysvětlování počítání strojových metrik
  nebo popisu, jak jsou porovnávány dva překlady jedné věty,
  jsou často použity termíny spojené se slovem \mbox{n-gram}.
Proto budou všechny tyto termíny v~této části vysvětleny,
  aby pozdější výklad byl jasný a nemuselo se odbíhat od tématu.

Posloupnost n po sobě jdoucích slov\footnote{
  Často se používá i výraz \textbf{token}. Token může být jak slovo, tak interpunkční znaménko a do \mbox{n-gramů} mohou patřit i interpunkční znaménka}
  se nazývá \textbf{\mbox{n-gram}}.
N-gramy,
  které se nacházejí i v~referenci i ve strojovém překladu,
  se v~rámci nástroje MT-ComparEval nazývají \textbf{potvrzené \mbox{n-gramy}}.
Potvrzené \mbox{n-gramy},
  které se při porovnávání dvou překladů nacházejí pouze v~jednom z~nich,
  jsou nazývány \textbf{zlepšující \mbox{n-gramy}}.
N-gramy,
  které nejsou potvrzené referencí
  a při porovnávání dvou překladů se nacházejí pouze v~jednom z~nich,
  jsou nazývany \textbf{zhoršující \mbox{n-gramy}}.
O~významu jednotlivých typů \mbox{n-gramů} je možné si udělat lepší představu na Obrázku \ref{img:n-grams}.

\begin{figure}

	\caption{
		Porovnání dvou překladů v~nástroji MT-ComparEval.
		Pastelovými odstíny žluté a modré barvy jsou zvýrazněny potvrzené \mbox{n-gramy},
		sytými odstíny žluté a modré barvy jsou zvýrazněny zlepšující \mbox{n-gramy}
		a červenou barvou jsou zvýrazněny zhoršující \mbox{n-gramy}.
	}
	\label{img:n-grams}
\end{figure}

