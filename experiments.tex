\chapter{Názvosloví \mbox{MT-ComparEval}}
\label{chap:experiments}

Pojmy \textbf{experiment}, \textbf{task}, \textbf{\mbox{n-gram}} (potvrzený, zlepšující, zhoršující) jsou důležité k~pochopení fungování celého nástroje \mbox{MT-ComparEval}.
Proto v~této kapitole bude podrobněji vysvětleno,
  co tyto pojmy znamenají a jaké jsou mezi nimi vztahy.

\section{Experiment}
Uživatelé mohou chtít testovat své překladače na různých textových doménách různých délek
  nebo testovat své překladače pro různé jazykové páry.
Nástroj \mbox{MT-ComparEval} proto umožňuje vytvořit různé \textbf{experimenty}.

Každý experiment obsahuje vlastní \textbf{zdroj} (věty, které mají být strojovým překladačem přeloženy) a
  \textbf{referenci} (člověkem přeložené věty, které budou později použity k~vyhodnocování jednotlivých strojových překladačů).
Pomocí různých dvojic zdrojů a referencí mohou uživatelé vyhodnocovat strojové překladače na různých testovacích sadách,
  což se jim může hodit v~případě,
  kdy se překladač na různých zdrojích chová různě.

\section{Task}
V~rámci experimetu mohou uživatelé porovnávat různé překlady zdrojových vět --
  ať už se jedná o~různé verze jednoho strojového překladače, nebo různé strojové překladače.
Aby mohl uživatel vyhodnocovat různé překlady,
  umožňuje nástroj \mbox{MT-ComparEval} nahrávat do experimentů tzv. \textbf{tasky}.
Každý task reprezentuje jednu verzi překladu zdrojových vět,
  která později může být porovnána s~jinou verzí překladu.
V~rámci této bakalářské práce budou strojové překlady nazývány zkráceně slovem \textbf{překlady}.

O~vytváření experimentů a tasků i o~dalších možnostech použití nástroje \mbox{MT-ComparEval} je možné se dozvědět v~\ref{chap:users}. kapitole.

\section{N-gramy -- potvrzené, zlepšující a zhoršující}
Při vysvětlování počítání strojových metrik
  nebo popisu, jak jsou porovnávány dva překlady jedné věty,
  jsou často použity termíny spojené se slovem \mbox{n-gram}.
Proto budou všechny tyto termíny v~této části vysvětleny,
  aby pozdější výklad byl jasný a nemuselo se odbíhat od tématu.

Posloupnost n po sobě jdoucích slov\footnote{
  V této práci se za slova považují i interpunkční znaménka, a \textbf{slovo} tedy znamená totéž co \textbf{token}.}
  se nazývá \textbf{\mbox{n-gram}}.
N-gramy,
  které se nacházejí i v~referenci i ve strojovém překladu,
  se v~rámci nástroje \mbox{MT-ComparEval} nazývají \textbf{kandidáti potvrzených \mbox{n-gramů}}.
Z těchto kandidátů je možné vybrat \textbf{potvrzené \mbox{n-gramy}}.
Ty jsou vybrány z kandidátů potvrzených \mbox{n-gramů} tak,
  aby počet výskytů daného potvrzeného \mbox{n-gramu} nebyl větší než počet výskytů daného \mbox{n-gramu} v~referenci.

Potvrzené \mbox{n-gramy},
  které se při porovnávání dvou překladů nacházejí pouze v~jednom z~nich,
  jsou nazývány \textbf{zlepšující \mbox{n-gramy}}.
N-gramy,
  které nejsou potvrzené referencí
  a při porovnávání dvou překladů se nacházejí pouze v~jednom z~nich,
  jsou nazývany \textbf{zhoršující \mbox{n-gramy}}.
O~významu jednotlivých typů \mbox{n-gramů} je možné si udělat lepší představu na Obrázku \ref{img:n-grams}.

V nástroji \mbox{MT-ComparEval} je možné nalézt i přehled \textbf{nejvíce zlepšujících \mbox{n-gramů}}.
Nejvíce zlepšující \mbox{n-gramy} jsou ty n-gramy,
  které byly nejčastěji prohlášeny za zlepšující.
Stejně tak je možné nalézt i přehled \textbf{nejvíce zhoršujících \mbox{n-gramů}},
  pro které platí totéž jako pro nejvíce zlepšující n-gramy.

To,
  že byl některý \mbox{n-gram} prohlášen za zlešující,
  neznaméná, že zlepšil kvalitu překladu.
Zlepšení je posuzováno pouze vzhledem k metrice BLEU.
Stejně tak i zhoršující n-gram nemusel zhoršit kvalitu překladu.

\begin{figure}
	\caption{
		Porovnání dvou překladů v~nástroji \mbox{MT-ComparEval}.
		Pastelovými odstíny žluté a modré barvy jsou zvýrazněny potvrzené \mbox{n-gramy},
		sytými odstíny žluté a modré barvy jsou zvýrazněny zlepšující \mbox{n-gramy}
		a červenou barvou jsou zvýrazněny zhoršující \mbox{n-gramy}.
	}
	\label{img:n-grams}
\end{figure}

\section{Diff dvou vět}
Nástroj \mbox{MT-ComparEval} umožňuje zobrazit \textbf{diff} dvou překladů anebo překladu a reference.
Pojmem diff dvou vět se rozumí rozdíl těchto vět.
Rozdíl vět je možné pro každou z těchto vět definovat jako slova,
  která se nachází v dané větě
  a zároveň se nenachází v nejdelší společné podposloupnosti porovnávaných vět

